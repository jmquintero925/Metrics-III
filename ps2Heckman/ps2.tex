\documentclass{article}

\usepackage[margin=1in]{geometry} 
\usepackage{amsmath,amsthm,amssymb,amsfonts, fancyhdr, color, comment, graphicx, environ,bbm}
\usepackage[dvipsnames]{xcolor}
\usepackage{subcaption}
\usepackage{mdframed}
\usepackage[shortlabels]{enumitem}
\usepackage{indentfirst}
\usepackage{hyperref}
\usepackage{placeins}
\usepackage{comment} %Comment large blocks
\usepackage{xfrac}
\usepackage{float} % To use "H" to force tables to be where wanted
\usepackage{booktabs} % Makes output from knittr:kable look better
\usepackage{parskip} %No indents for paragraphs
\usepackage[shortlabels]{enumitem} % To enumerate with letters
\hypersetup{
 colorlinks=true,
 linkcolor=blue,
 filecolor=magenta, 
 urlcolor=blue,
}
\pagestyle{fancy}
\usepackage{todonotes}

\newcommand{\E}{\mathbb{E}}
\renewcommand{\H}{\mathcal{H}}
\renewcommand{\L}{\mathcal{L}}
\newcommand{\dU}[1]{\ensuremath\frac{\partial u}{\partial #1}}
\newcommand{\dV}[1]{\ensuremath\frac{\partial v}{\partial #1}}
\newcommand{\ppx}[2]{\ensuremath\frac{\partial #1}{\partial #2}}
\newcommand{\ddx}[2]{\ensuremath\frac{d #1}{d #2}}
\newcommand{\indp}{\perp\!\!\!\perp} 

%\usepackage{mathpazo} %Dylan's fancy math bullshit that I don't want
\usepackage{microtype}
\usepackage{graphicx}
\usepackage{setspace}

%Footnote without a number
\newcommand\blfootnote[1]{%
  \renewcommand\thefootnote{}\footnote{#1}%
  \addtocounter{footnote}{-1}%
}

% Problem formatting [Alex]

\newenvironment{problem}[1]
    { \begin{mdframed}[backgroundcolor=Periwinkle!20] \textbf{(#1)} }
    {  \end{mdframed}}
% Define solution environment
\newenvironment{solution}{\textbf{Solution}\\}

%%%%%%%%%%%%%%%%%%%%%%%%%%%%%%%%%%%%%%%%%%%%%
%Fill in the appropriate information below
\lhead{Problem Set 2}
\rhead{Empirical Analysis} 
\title{Problem Set 2}
\author{Alex Weinberg \and Isaac Norwich \and Jose M. Quintero}

%%%%%%%%%%%%%%%%%%%%%%%%%%%%%%%%%%%%%%%%%%%%%
\begin{document}
\maketitle

\begin{comment}
Q3 - 3 parts - no solutions - Jose
Q4 - 1 part  - no solutions - Alex
Q5 - 1 part  - no solutions - Jose
Q8 - 5 parts - no solutions - Isaac

Q1 - 7 parts - solutions - Isaac & Jose
Q2 - 4 parts - solutions - Jose
Q6 - 1 part  - solutions - Alex
Q7 - 3 parts - solutions - Alex

Parts:
Isaac - 5
Alex
Jose

Total of 21 parts and 2 other questions

\end{comment}


Our code can be found in this GitHub respository: \url{https://github.com/jmquintero925/Metrics-III/tree/main/ps2Heckman}


%%%%%%%%%%%%%%%%%%%%%%%%%%%%%%%%%% QUESTION 1 %%%%%%%%%%%%%%%%%%%%%%%%%%%%%%%%%
\section*{Problem 1}
Answer the questions embedded in the econometric causality model handouts based on Heckman (2008).

\begin{problem}{Question 1 Slide 27}
Question: Can agents ex ante evaluate the ex post evaluation?
\end{problem}
\begin{solution}
\end{solution}


\begin{problem}{Question 2 Slide 32}
Question: How can agents identify what might have been for states they have not experienced? Consider alternative approaches.
\end{problem}
\begin{solution}
\end{solution}
 
\begin{problem}{Question 3 Slide 35}
Question: What are the precise requirements for solving P3 for the PRTE?
\end{problem}
\begin{solution}
\end{solution}
 
 
\begin{problem}{Question 4 Slide 36}
Question: In the context of a policy of tuition reduction,
under what conditions is Ya = Yb;Ya = Yb where Yj denotes 0011i
the present value of life cycle earnings under policy j in state i?
\end{problem}
\begin{solution}
\end{solution}

\begin{problem}{Question 5 Slide 38}
Question: What is the relationship between PRTE and ITT (Intention To Treat)? Is PRTE a causal parameter?
\end{problem}
\begin{solution}
\end{solution}


\begin{problem}{Question 6 Slide 41}
Question: Is LATE a causal parameter? How does it address P1-P3?
\end{problem}
\begin{solution}
\end{solution}


\begin{problem}{Question 7 Slide 43}
Question: Verify each claim in this box.
\end{problem}
\begin{solution}
\end{solution}



\newpage
%%%%%%%%%%%%%%%%%%%%%%%%%%%%%%%%%% QUESTION 2 %%%%%%%%%%%%%%%%%%%%%%%%%%%%%%%%%
\section*{Problem 2}
Answer the questions embedded in the ``Classical Discrete Choice Theory'' handout.

\begin{problem}{Question 1 Slide 17}
Prove
\end{problem}
\begin{solution}
\end{solution}

\begin{problem}{Question 2 Slide 28}
Prove why introduction of identical good changes probability of riding a bus.
\end{problem}
\begin{solution}
Hola hola!
\end{solution}


\begin{problem}{Question 1 Slide 40}
Prove this
\end{problem}
\begin{solution}
\end{solution}

\begin{problem}{Question 2 Slide 56}
Prove it can be used to identify $\sigma_{U}^{2}$ and $\sum_{\beta}$.
\end{problem}
\begin{solution}
\end{solution}

\newpage

%%%%%%%%%%%%%%%%%%%%%%%%%%%%%%%%%% QUESTION 3 %%%%%%%%%%%%%%%%%%%%%%%%%%%%%%%%%
\section*{Problem 3}
 For the model $Y=X_{1} \beta_{1}+X_{2} \beta_{2}+U$,
\begin{align*}
\begin{aligned}
&E\left(U \mid X_{1}, X_{2}\right)=0 \\
&\sum_{X_{1}, X_{2}} \text { full rank, }
\end{aligned}
\end{align*}
discuss and compare the properties of the three estimators: \\
\begin{align*}
    &(a) \quad \text{ OLS } \beta_{1} \\
    &(b) \quad \hat{\beta}_{1} \text{ from a regression of $Y$ on $X_{1}$ alone.} \\
    &(c) \quad \hat{\beta}_{1}= 
        \begin{cases}
            \beta_{1} \mathrm{OLS}, & \text { if } t_{\hat{\beta}_{1}} \geq 2 \\ 
            \hat{\beta}_{1}, & \text{ otherwise (from a regression of } Y \text{ on } X_{1} \text { alone). }
        \end{cases}    
\end{align*}

\begin{solution}
\end{solution}

\newpage

%%%%%%%%%%%%%%%%%%%%%%%%%%%%%%%%%% QUESTION 4 %%%%%%%%%%%%%%%%%%%%%%%%%%%%%%%%%
\section*{Problem 4}
Answer the questions embedded in the ``Hypothesis Testing: Part I'' handout.

\todo[inline]{I have no idea where the questions are in these slides -Isaac}

\begin{problem}{Question 1 Slide 7}
How to construct a ‘best’ test? Compare alternative tests.
Any monotonic transformation of the “$t$” statistic produces
the same $P$ value.
\end{problem}
\begin{solution}
\end{solution}

\begin{problem}{Question 2 Slide 7}
Pure significance tests depend on the sampling rule used to
collect the data. This is not necessarily bad.
\end{problem}
\begin{solution}
\end{solution}

\newpage

%%%%%%%%%%%%%%%%%%%%%%%%%%%%%%%%%% QUESTION 5 %%%%%%%%%%%%%%%%%%%%%%%%%%%%%%%%%
\section*{Problem 5}
Answer the questions embedded in the ``How to Correct for Sampling Biases'' handout.

\begin{problem}{Question 1 Slide 5}
Why?
\end{problem}
\begin{solution}
\end{solution}

\newpage

%%%%%%%%%%%%%%%%%%%%%%%%%%%%%%%%%% QUESTION 6 %%%%%%%%%%%%%%%%%%%%%%%%%%%%%%%%%
\section*{Problem 6}
Answer the questions embedded in the ``Roy Models of Policy Evaluation'' handout.
\todo[inline]{I don't think we have this yet, but its just 1 part from Q1B from last year. -Isaac}

\begin{problem}{Question 1 Slide 9} How does this covariance relate to the question of whether a country is a meritocracy?
\end{problem}
\begin{solution}
\end{solution}


\newpage

%%%%%%%%%%%%%%%%%%%%%%%%%%%%%%%%%% QUESTION 7 %%%%%%%%%%%%%%%%%%%%%%%%%%%%%%%%%
\section*{Problem 7}
Answer the questions embedded in the ``Notes on Identification of the Roy Model'' and the ``Generalized Roy Model'' handout.

%%%%%``Notes on Identification of the Roy Model''
\begin{problem}{Question 1 Slide 5}
Just invert known $f_{U_{l}}$ to establish $\frac{\mu_{l}(X, Z)}{\sigma_{l}}$. Prove.
\end{problem}
\begin{solution}
\end{solution}

\begin{problem}{Question 1 Slide 10}
Problem: Prove this using the first line of $(* *)$ realizing that you know $\frac{U_{1}}{\delta_{I}}$.
\end{problem}
\begin{solution}
\end{solution}

%%%%% ``Generalized Roy Model''
\begin{problem}{Question 1 Slide 31}
Prove MTE $=\frac{\partial E(Y \mid Z=z)}{\partial P(z)}$
\end{problem}
\begin{solution}
\end{solution}

\newpage

%%%%%%%%%%%%%%%%%%%%%%%%%%%%%%%%%% QUESTION 8 %%%%%%%%%%%%%%%%%%%%%%%%%%%%%%%%%
\section*{Problem 8}
 
\begin{problem}{a}
\end{problem}
\begin{solution}
\end{solution}


\end{document}
