\documentclass{article}
%\documentclass[12pt]{article} % Put this in the actual tex file or overleaf throws a hissy fit
\usepackage[margin=1in]{geometry} 
\usepackage{amsmath,amsthm,amssymb,amsfonts, fancyhdr, color, comment, graphicx, environ}
\usepackage{xcolor}
\usepackage{subcaption}
\usepackage{mdframed}
\usepackage[shortlabels]{enumitem}
\usepackage{indentfirst}
\usepackage{hyperref}
\usepackage{placeins}
\usepackage{comment} %Comment large blocks
\usepackage{xfrac}
\usepackage{parskip} %No indents for paragraphs
\usepackage[shortlabels]{enumitem} % To enumerate with letters
\hypersetup{
 colorlinks=true,
 linkcolor=blue,
 filecolor=magenta, 
 urlcolor=blue,
}
\pagestyle{fancy}
\usepackage{todonotes}

\newcommand{\E}{\mathbb{E}}
\renewcommand{\H}{\mathcal{H}}
\renewcommand{\L}{\mathcal{L}}
\newcommand{\dU}[1]{\ensuremath\frac{\partial u}{\partial #1}}
\newcommand{\dV}[1]{\ensuremath\frac{\partial v}{\partial #1}}
\newcommand{\ppx}[2]{\ensuremath\frac{\partial #1}{\partial #2}}
\newcommand{\ddx}[2]{\ensuremath\frac{d #1}{d #2}}

%\usepackage{mathpazo} %Dylan's fancy math bullshit that I don't want
\usepackage{microtype}
\usepackage{graphicx}
\usepackage{setspace}

%Footnote without a number
\newcommand\blfootnote[1]{%
  \renewcommand\thefootnote{}\footnote{#1}%
  \addtocounter{footnote}{-1}%
}

%%%%%%%%%%%%%%%%%%%%%%
% Set up fancy header/footer
\pagestyle{fancy}
\fancyhead[LO,L]{Isaac Norwich}
\fancyhead[CO,C]{Theory of Income I - Problem Set #}
\fancyhead[RO,R]{\today}
\fancyfoot[LO,L]{}
\fancyfoot[CO,C]{\thepage}
\fancyfoot[RO,R]{}
\renewcommand{\headrulewidth}{0.4pt}
\renewcommand{\footrulewidth}{0.4pt}

%%%%%%%%%%%%%%%%%%%%%%%%%%%%%%%%%%%%%%%%%%%%%
%Fill in the appropriate information below
\lhead{Problem Set 2}
\rhead{Empirical Analysis} 
\title{Problem Set 2}
\author{Alex Weinberg \and Isaac Norwich \and Jose M. Quintero}

%%%%%%%%%%%%%%%%%%%%%%%%%%%%%%%%%%%%%%%%%%%%%
\begin{document}
\maketitle

\begin{comment}
Q3 - 3 parts - no solutions - Jose
Q4 - 1 part  - no solutions - Alex
Q5 - 1 part  - no solutions - Jose
Q8 - 5 parts - no solutions - Isaac

Q1 - 7 parts - solutions - Isaac & Jose
Q2 - 4 parts - solutions - Jose
Q6 - 1 part  - solutions - Alex
Q7 - 3 parts - solutions - Alex

Parts:
Isaac - 5
Alex
Jose

Total of 21 parts and 2 other questions

\end{comment}


Our code can be found in this GitHub respository: \url{https://github.com/jmquintero925/Metrics-III/tree/main/ps2Heckman}


%%%%%%%%%%%%%%%%%%%%%%%%%%%%%%%%%% QUESTION 1 %%%%%%%%%%%%%%%%%%%%%%%%%%%%%%%%%
\section*{Problem 1}
Answer the questions embedded in the econometric causality model handouts based on Heckman (2008).

\begin{problem}{Question 1 Slide 27}
Question: Can agents ex ante evaluate the ex post evaluation?
\end{problem}
\begin{solution}
\end{solution}


\begin{problem}{Question 2 Slide 32}
Question: How can agents identify what might have been for states they have not experienced? Consider alternative approaches.
\end{problem}
\begin{solution}
\end{solution}
 
\begin{problem}{Question 3 Slide 35}
Question: What are the precise requirements for solving P3 for the PRTE?
\end{problem}
\begin{solution}
\end{solution}
 
 
\begin{problem}{Question 4 Slide 36}
Question: In the context of a policy of tuition reduction,
under what conditions is Ya = Yb;Ya = Yb where Yj denotes 0011i
the present value of life cycle earnings under policy j in state i?
\end{problem}
\begin{solution}
\end{solution}

\begin{problem}{Question 5 Slide 38}
Question: What is the relationship between PRTE and ITT (Intention To Treat)? Is PRTE a causal parameter?
\end{problem}
\begin{solution}
\end{solution}


\begin{problem}{Question 6 Slide 41}
Question: Is LATE a causal parameter? How does it address P1-P3?
\end{problem}
\begin{solution}
\end{solution}


\begin{problem}{Question 7 Slide 43}
Question: Verify each claim in this box.
\end{problem}
\begin{solution}
\end{solution}



\newpage
%%%%%%%%%%%%%%%%%%%%%%%%%%%%%%%%%% QUESTION 2 %%%%%%%%%%%%%%%%%%%%%%%%%%%%%%%%%
\section*{Problem 2}
Answer the questions embedded in the ``Classical Discrete Choice Theory'' handout.

\begin{problem}{Question 1 Slide 17}
Prove
\end{problem}
\begin{solution}
Let $F(\varepsilon_1,...,\varepsilon_N)$ the join CDF of the idiosyncratic shocks. Then note that
\begin{align*}
    \ppx{}{\varepsilon_j}F(\varepsilon_1,...,\varepsilon_N) &= \ppx{}{\varepsilon_j} \int_{-\infty}^{\varepsilon_1}\cdots\int_{-\infty}^{\varepsilon_j}\cdots\int_{-\infty}^{\varepsilon_N} f(x_1,...,x_N)\mathrm{d}x_1\cdots\mathrm{d}x_j\cdots\mathrm{d}x_N \\
     &= \int_{-\infty}^{\varepsilon_1}\cdots\int_{-\infty}^{\varepsilon_{j-1}}\int_{-\infty}^{\varepsilon_{j+1}}\cdots\int_{-\infty}^{\varepsilon_N} f(x_1,...,\varepsilon_j,...,x_N)\mathrm{d}x_1\cdots\mathrm{d}x_{j-1}\mathrm{d}x_{j+1}\cdots\mathrm{d}x_N
\end{align*}
where the second line uses Fubini's theorem (interchange the order of integrals) and the fact for pdfs, the integral and the derivative are commutative operations. Using this fact, note that the conditional CDF can be written as 
\begin{align*}
    F(\varepsilon_1,...\varepsilon_{j-1},\varepsilon_{j+1},...,\varepsilon_N\vert\varepsilon_j) &= \frac{1}{f_{\varepsilon_j}(\varepsilon_j)} \int_{-\infty}^{\varepsilon_1}\cdots\int_{-\infty}^{\varepsilon_{j-1}}\int_{-\infty}^{\varepsilon_{j+1}}\cdots\int_{-\infty}^{\varepsilon_N} f(x_1,...,\varepsilon_j,...,x_N)\mathrm{d}x_1\cdots\mathrm{d}x_{j-1}\mathrm{d}x_{j+1}\cdots\mathrm{d}x_N \\ 
    &=\frac{1}{f_{\varepsilon_j}(\varepsilon_j)} \ppx{}{\varepsilon_j}F(\varepsilon_1,...,\varepsilon_N)
\end{align*}
Using this equation, we can prove the result by taking conditional expectations. 
\begin{align*}
\Pr\left(\varepsilon_{I}\leq v_{j}-v_{I} + \varepsilon_{j}, \quad \forall I \neq j\right)=&\E\left[\Pr\left(\varepsilon_{I}\leq v_{j}-v_{I} + \varepsilon_{j}, \forall I \neq j\Big\vert \varepsilon_{j}\right)\right]\\
%%%%%%%
&\int_{-\infty}^{\infty} F\left(v_{j}-v_{1} + \varepsilon_{1},..., v_{j}-v_{j-1} + \varepsilon_{j},\varepsilon_{j},...v_{j}-v_{N} + \varepsilon_{j}\right) f_{\varepsilon_j}\left(\varepsilon_{j}\right) d \varepsilon_{j}\\
%%%%%%%%
&\int_{-\infty}^{\infty} \frac{\partial F}{\partial \varepsilon_{j}}\left(v_{j}-v_{1} + \varepsilon_{1},..., v_{j}-v_{j-1} + \varepsilon_{j},\varepsilon_{j},...v_{j}-v_{N} + \varepsilon_{j}\right) d \varepsilon_{j}
\end{align*}
\end{solution}

\begin{problem}{Question 2 Slide 28}
Prove why introduction of identical good changes probability of riding a bus.
\end{problem}
\begin{solution}
Hola hola!
\end{solution}


\begin{problem}{Question 1 Slide 40}
Prove this
\end{problem}
\begin{solution}
\end{solution}

\begin{problem}{Question 2 Slide 56}
Prove it can be used to identify $\sigma_{U}^{2}$ and $\sum_{\beta}$.
\end{problem}
\begin{solution}
\end{solution}

\newpage

%%%%%%%%%%%%%%%%%%%%%%%%%%%%%%%%%% QUESTION 3 %%%%%%%%%%%%%%%%%%%%%%%%%%%%%%%%%
\section*{Problem 3}
 For the model $Y=X_{1} \beta_{1}+X_{2} \beta_{2}+U$,
\begin{align*}
\begin{aligned}
&E\left(U \mid X_{1}, X_{2}\right)=0 \\
&\sum_{X_{1}, X_{2}} \text { full rank, }
\end{aligned}
\end{align*}
discuss and compare the properties of the three estimators: \\
\begin{enumerate}[label=(\alph*)]
    \item OLS $\beta_1$
    \item $\hat{\beta_1}$
    \item $\hat{\beta}_1=\begin{cases}
            \beta_{1} \mathrm{OLS}, & \text { if } t_{\hat{\beta}_{1}} \geq 2 \\ 
            \hat{\beta}_{1}, & \text{ otherwise (from a regression of } Y \text{ on } X_{1} \text { alone). }
        \end{cases}    $
\end{enumerate}



\begin{solution}
For each of the cases
\begin{enumerate}[label=(\alph*)]
    \item $\beta_1$ from OLS is consistent and unbiased. It is not necessarily efficient though as no assumptions are given on the structure of the error terms. In the absence of heteroscedasticity then the problem would satisfy Gauss-Markov hypothesis and it will have minimal variance over the class of linear unbiased estimators. 
    \item Consider the formula for omitted variables
    \begin{align*}
        \hat{\beta}_1 &= \frac{\cov(Y,X_1)}{\var(X_1)} \\ 
        &= \frac{\cov(X_1\beta_1+X_2\beta_2+U,X_1)}{\var(X_1)} \\ 
        &= \beta_1 + \beta_2\frac{\cov(X_1,X_2)}{\var(X_1)}
    \end{align*}
    Then the properties of this estimator are going to depend on the covariance between $X_1$ and $X_2$. If $\cov(X_1,X_2)=0$ then the estimator will be both consistent and unbiased. Otherwise it will both inconsistent and biased. 
    \item 
\end{enumerate}

\end{solution}

\newpage

%%%%%%%%%%%%%%%%%%%%%%%%%%%%%%%%%% QUESTION 4 %%%%%%%%%%%%%%%%%%%%%%%%%%%%%%%%%
\section*{Problem 4}
Answer the questions embedded in the ``Hypothesis Testing: Part I'' handout.

\begin{problem}{Slide 6}
Prove that for random variable with density (absolutely continuous with
Lebesgue measure) Z = FX (X ) is uniform for any X given
that FX is continuous.
\end{problem}

\begin{solution}
Given any random continuous variable X, define $Z := f(X) = F_X(X)$. Because its a probability it is in the unit interval. Define $G$ as the CDF of $Z$. Assuming requisite continuity:

\begin{align*}
G(z) &= \operatorname{P}(Z\leq z) \\
        &= \operatorname{P}(f(X)\leq z) \\
        &= \operatorname{P}(X\leq f^{-1}(z)) \\
        &= f (f^{-1}(z)) \\
        &= z
\end{align*}

QED.

\end{solution}

% ---------------------
\begin{problem}{Slide 12}
\end{problem}

Increasing the sample size increases the power of the test to detect smaller differences between the null and alternative hypotheses. We reject the null hypothesis $\mathrm{H}_{0}$ if

$$
|\bar{X}-\mu|>c(\alpha)=\frac{\sigma}{\sqrt{T}} \Phi^{-1}(\alpha)
$$

As $T \rightarrow \infty$, the critical value converges to zero almost surely, which implies that we always reject the null.
% ---------------------
\begin{problem}{Slide 29}
\end{problem}

With two trials, we get $C\left(\mathrm{X}_{1}, \mathrm{X}_{2}\right)=\theta_{0}$ in 3 of the 4 possible combinations.



% ---------------------
\begin{problem}{Slide 30}
\end{problem}
If the two labs are exactly the same, the coin toss need not be included, as the resulting marginal density under any hypothesis will be the same regardless of whether the coin toss is included. However, if the two labs are different in their testing procedures, when a single confidence interval must be chosen ex-ante to conduct the statistical test, then this should account for the coin flip in the testing statistic.


% ---------------------
\begin{problem}{Slide 39}
\end{problem}
The probability of getting any sequence of $x$ black and 3 red balls (without a stopping rule) is given by $\theta^{x}(1-\theta)^{3}$. The sequence stops after three red balls are drawn. This implies that the last draw is red. This leaves $\left(\begin{array}{c}x+2 \\ x\end{array}\right)$ possible arrangements of the remaining $x+2$ balls.

% ---------------------
\begin{problem}{Slide 72}
\end{problem}

Slide 72: An OLS derivation.

\newpage

%%%%%%%%%%%%%%%%%%%%%%%%%%%%%%%%%% QUESTION 5 %%%%%%%%%%%%%%%%%%%%%%%%%%%%%%%%%
\section*{Problem 5}
Answer the questions embedded in the ``How to Correct for Sampling Biases'' handout.

\begin{problem}{Question 1 Slide 5}
Why?
\end{problem}
\begin{solution}
Consider the censored model where the reseacher observes $y_i$ according to the following rule 
\begin{equation*}
    y_i = \begin{cases}
        y_0 &\mbox{if } y^*_i< y_0 \\ 
        y_i^* &\mbox{if } y^*_i\geq y_0
    \end{cases}
\end{equation*}
Assume that the underlying random utility has the following functional form
\begin{equation*}
 y_i^* = X_i'\beta + u_i
\end{equation*}
with $u_i\sim\mathcal{N}(0,\sigma_u^2)$. Then the density of $y_i$ is 
\begin{align*}
    g(y_i) &= \Pr(y_i=0) + f(y_i\vert y_i=y_i^*)\Pr(y_i = y_i^*) \\ 
    &= \Pr(y_i^*<y_0) + f(y_i^*\vert y_i^*\geq y_0)\Pr(y_i^*\geq y_0) \\
    &= \Pr\left(\frac{u_i}{\sigma_u}<\frac{y_0-X_i\beta}{\sigma_u}\right) + f(y_i^*\vert y_i^*\geq y_0)\Pr\left(\frac{u_i}{\sigma_u}\geq\frac{y_0-X_i\beta}{\sigma_u}\right) \\ 
    &= \Phi\left(\frac{y_0-X_i\beta}{\sigma_u}\right) + f(y_i^*\vert y_i^*\geq y_0)\left[1-\Phi\left(\frac{y_0-X_i\beta}{\sigma_u}\right)\right]
\end{align*}
Note that this itself is a density. Before proving that claim, lets calculate the conditional density:
\begin{align*}
    f(y_i\vert y_i^*\geq y_0) &= \frac{1}{\Pr\left(y_i^*\geq y_0\right)}f(y_i^*) \\ 
    &= \frac{1}{1-\Phi\left(\frac{y_0-X_i\beta}{\sigma_u}\right)} \frac{1}{\sigma_u\sqrt{2\pi}} \exp\left(-\frac{1}{2}\left(\frac{y_i-X_i'\beta}{\sigma_u}\right)^2\right) \tag{Let $x_i=\frac{y_i-X_i'\beta}{\sigma_u}$} \\ 
    &= \frac{1}{1-\Phi\left(\frac{y_0-X_i\beta}{\sigma_u}\right)} \frac{1}{\sigma_u\sqrt{2\pi}} \exp\left(-\frac{1}{2}x_i^2\right) \\
     &= \frac{\frac{1}{\sigma_u}\phi(x_i)}{1-\Phi\left(\frac{y_0-X_i\beta}{\sigma_u}\right)}
     = \frac{\frac{1}{\sigma_u}\phi\left(\frac{y_i-X_i'\beta}{\sigma_u}\right)}{1-\Phi\left(\frac{y_0-X_i\beta}{\sigma_u}\right)}
\end{align*}
Putting everything together the density of $y_i$ is 
\begin{equation}
    g(y_i) = \Phi\left(\frac{y_0-X_i\beta}{\sigma_u}\right) + \frac{1}{\sigma_u}\phi\left(\frac{y_i-X_i'\beta}{\sigma_u}\right)
\end{equation}
\textcolor{red}{Check the density}
\end{solution}

\newpage

%%%%%%%%%%%%%%%%%%%%%%%%%%%%%%%%%% QUESTION 6 %%%%%%%%%%%%%%%%%%%%%%%%%%%%%%%%%
\section*{Problem 6}
Answer the questions embedded in the ``Roy Models of Policy Evaluation'' handout.
\todo[inline]{I don't think we have this yet, but its just 1 part from Q1B from last year. -Isaac}

\begin{problem}{Question 1 Slide 9} How does this covariance relate to the question of whether a country is a meritocracy?
\end{problem}
\begin{solution}
\end{solution}


\newpage

%%%%%%%%%%%%%%%%%%%%%%%%%%%%%%%%%% QUESTION 7 %%%%%%%%%%%%%%%%%%%%%%%%%%%%%%%%%
\section*{Problem 7}
Answer the questions embedded in the ``Notes on Identification of the Roy Model'' and the ``Generalized Roy Model'' handout.

%%%%%``Notes on Identification of the Roy Model''
\begin{problem}{Question 1 Slide 5}
Just invert known $f_{U_{l}}$ to establish $\frac{\mu_{l}(X, Z)}{\sigma_{l}}$. Prove.
\end{problem}
\begin{solution}
\end{solution}

\begin{problem}{Question 1 Slide 10}
Problem: Prove this using the first line of $(* *)$ realizing that you know $\frac{U_{1}}{\delta_{I}}$.
\end{problem}
\begin{solution}
\end{solution}

%%%%% ``Generalized Roy Model''
\begin{problem}{Question 1 Slide 31}
Prove MTE $=\frac{\partial E(Y \mid Z=z)}{\partial P(z)}$
\end{problem}
\begin{solution}
\end{solution}

\newpage

%%%%%%%%%%%%%%%%%%%%%%%%%%%%%%%%%% QUESTION 8 %%%%%%%%%%%%%%%%%%%%%%%%%%%%%%%%%
\section*{Problem 8}
 
\begin{problem}{a}
\end{problem}
\begin{solution}
Dataset
1 pos corr
2 neg corr
3 no corr
4 as Z increases so does prob
5 as Z increases so does prob

Ok so part A is to do a probit of d given x
Treat each dataset separately

Part B vs Part C???
Look at references at end of Filippo's stuff


\end{solution}



\end{document}
