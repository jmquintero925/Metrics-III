\documentclass{article}
%\documentclass[12pt]{article} % Put this in the actual tex file or overleaf throws a hissy fit
\usepackage[margin=1in]{geometry} 
\usepackage{amsmath,amsthm,amssymb,amsfonts, fancyhdr, color, comment, graphicx, environ}
\usepackage{xcolor}
\usepackage{subcaption}
\usepackage{mdframed}
\usepackage[shortlabels]{enumitem}
\usepackage{indentfirst}
\usepackage{hyperref}
\usepackage{placeins}
\usepackage{comment} %Comment large blocks
\usepackage{xfrac}
\usepackage{parskip} %No indents for paragraphs
\usepackage[shortlabels]{enumitem} % To enumerate with letters
\hypersetup{
 colorlinks=true,
 linkcolor=blue,
 filecolor=magenta, 
 urlcolor=blue,
}
\pagestyle{fancy}
\usepackage{todonotes}

\newcommand{\E}{\mathbb{E}}
\renewcommand{\H}{\mathcal{H}}
\renewcommand{\L}{\mathcal{L}}
\newcommand{\dU}[1]{\ensuremath\frac{\partial u}{\partial #1}}
\newcommand{\dV}[1]{\ensuremath\frac{\partial v}{\partial #1}}
\newcommand{\ppx}[2]{\ensuremath\frac{\partial #1}{\partial #2}}
\newcommand{\ddx}[2]{\ensuremath\frac{d #1}{d #2}}

%\usepackage{mathpazo} %Dylan's fancy math bullshit that I don't want
\usepackage{microtype}
\usepackage{graphicx}
\usepackage{setspace}

%Footnote without a number
\newcommand\blfootnote[1]{%
  \renewcommand\thefootnote{}\footnote{#1}%
  \addtocounter{footnote}{-1}%
}

%%%%%%%%%%%%%%%%%%%%%%
% Set up fancy header/footer
\pagestyle{fancy}
\fancyhead[LO,L]{Isaac Norwich}
\fancyhead[CO,C]{Theory of Income I - Problem Set #}
\fancyhead[RO,R]{\today}
\fancyfoot[LO,L]{}
\fancyfoot[CO,C]{\thepage}
\fancyfoot[RO,R]{}
\renewcommand{\headrulewidth}{0.4pt}
\renewcommand{\footrulewidth}{0.4pt}

%%%%%%%%%%%%%%%%%%%%%%%%%%%%%%%%%%%%%%%%%%%%%
%Fill in the appropriate information below
\lhead{Alex Weinberg, Isaac Norwich, José Quintero}
\rhead{Empirical Analysis III} 
\chead{Problem Set 1}

%%%%%%%%%%%%%%%%%%%%%%%%%%%%%%%%%%%%%%%%%%%%%
\begin{document}

%%%%%%%%%%%%%%%%%%%%%%%%%%%%%%%%%% QUESTION 1 %%%%%%%%%%%%%%%%%%%%%%%%%%%%%%%%%
\section*{Problem 1}

IID sample. $Y_{i}$ is future earnings. $D_{i}\in\left\{ 0,1\right\} $
is a binary treatment - recieving scholarship. 

\begin{problem}{1}
(a) Compute $\left(\beta_{0}, \beta_{1, i}, U_{i}\right)$ so that $Y_{i}=\beta_{0}+\beta_{1, i} D_{i}+U_{i}$, where $E\left(U_{i}\right)=0$. How would you interpret the random coefficient $\beta_{1, i}$ ? Is it identified for any individual $i$ ? Explain. 
\end{problem}

Recall that we can manipulate the definition of the outcome variable.
\begin{align*}
Y_{i} & =Y_{1i}D_{i}+Y_{0i}\left(1-D_{i}\right)\\
 & =Y_{0i}+D_{i}\left(Y_{1i}-Y_{0i}\right)
\end{align*}

So then I can define
\begin{align*}
\beta_{1,i} & =Y_{1i}-Y_{0i}\\
\beta_{0} & =E\left[Y_{0,i}\right]\\
U_{i} & =Y_{0,i}-E\left[Y_{0,i}\right]
\end{align*}

note that $E\left[U_{i}\right]=0$ by defintion. Finally stack.

\[
\ensuremath{Y_{i}=\beta_{0}+\beta_{1,i}D_{i}+U_{i}}
\]

The interpretation of $\beta_{i1}$ is the \emph{individual treatment
effect. }Note that we cannot recover $\beta_{i1}$ because we cannot
disentangle the individual elements $\beta_{1,i}$ and $U_{i}$. However
we can recover averages. 

\begin{problem}{b}
Is $\kappa=E\left(Y_{i} \mid D_{i}=1\right)-E\left(Y_{i} \mid D_{i}=0\right)$ identified? When is $\kappa$ equal to ATT, ATUT, or ATE? \end{problem}

Yes $\kappa$ is identified. We observe
\begin{align*}
E\left[Y_{i}|D_{i}=1\right] & =E\left[Y_{1i}D_{i}+Y_{0i}\left(1-D_{i}\right)|D_{i}=1\right]\\
 & =E\left[Y_{1i}|D_{i}=1\right]
\end{align*}

We also observe
\begin{align*}
E\left[Y_{i}|D_{i}=0\right] & =E\left[Y_{1i}D_{i}+Y_{0i}\left(1-D_{i}\right)|D_{i}=0\right]\\
 & =E\left[Y_{0i}|D_{i}=0\right]
\end{align*}

So our parameter
\begin{align*}
\kappa & :=E\left[Y_{i}|D_{i}=1\right]-E\left[Y_{i}|D_{i}=0\right]\\
 & =E\left[Y_{1i}|D_{i}=1\right]-E\left[Y_{0i}|D_{i}=0\right]
\end{align*}

If treatment is uncorrelated with potential outcomes then then our
parameter is equal to the ATE
\begin{align*}
\\
\kappa & :=E\left[Y_{i}|D_{i}=1\right]-E\left[Y_{i}|D_{i}=0\right]\\
 & =E\left[Y_{1i}|D_{i}=1\right]-E\left[Y_{0i}|D_{i}=0\right]\\
 & =E\left[Y_{1i}\right]-E\left[Y_{0i}\right]\tag{\text{because \ensuremath{Y_{1i},Y_{0i}}\ensuremath{\perp D_{i}}}}\\
 & =E\left[Y_{1i}-Y_{0i}\right]\tag{\text{by linearity of expectations}}
\end{align*}

If treatment is always uncorrelated with potential outcomes then $\kappa=ATE=ATUT=ATT$.
But if treatment usually is selected into then the ATE will not necessarily
equal the ATUT, ATT. I suppose the way to figure out the ATT would
be to randomly assign the scholarship/control among students who apply.
ATUT could be assessed by randomly assigning scholarship/control among
students who do not apply to the scholarship. \begin{problem}{c}Suppose the scholarship is only given to high-achieving students, who are already more likely to have higher earnings. Will $\kappa$ overstate or understate the ATT and the ATUT?
\end{problem}

Our observed parameter
\begin{align*}
\kappa & :=E\left[Y_{1i}|D_{i}=1\right]-E\left[Y_{0i}|D_{i}=0\right]\\
 & =\underbrace{E\left[Y_{1i}|D_{i}=1\right]-E\left[Y_{0i}|D_{i}=1\right]}_{\text{ATT}}+\underbrace{E\left[Y_{0i}|D_{i}=1\right]-E\left[Y_{0i}|D_{i}=0\right]}_{\text{selection bias}}
\end{align*}

So if the scholarship is givent to high achieving students - who have
high $Y_{0i}$ then selection bias will be positive. This means that
$\kappa$ overstates the ATT. Similarly for ATUT,
\begin{align*}
\kappa & :=E\left[Y_{1i}|D_{i}=1\right]-E\left[Y_{0i}|D_{i}=0\right]\\
 & =\underbrace{E\left[Y_{1i}|D_{i}=1\right]-E\left[Y_{1i}|D_{i}=0\right]}_{\text{selection}}+\underbrace{E\left[Y_{1i}|D_{i}=0\right]-E\left[Y_{0i}|D_{i}=0\right]}_{\text{ATUT}}
\end{align*}

Assuming the selection is still positive, this means that $\kappa>ATUT$.\begin{problem}{d} Suppose the scholarship is given based on financial need, so that its recipients are more positively affected by $D_{i}$ than other students are. How do you expect the ATT, ATUT, and ATE will compare in magnitude? Provide intuition behind your comparisons.
\end{problem}

Suppose now that the scholarship is targeted towards students most
likely to benefit from the program. In other words,

\[
E\left[Y_{1i}-Y_{0i}|D_{i}=1\right]>E\left[Y_{1i}-Y_{0i}|D_{i}=0\right]
\]

Notice here that the ATT is larger that the ATUT. Since the ATE is
a weighted average of the ATT and ATUT we get
\[
ATT>ATE>ATUT
\]

The intuition is obvious, if the program targets those most likely
to benefit then the treatment effect of the program will be larger
than the treatment effect for a student randomly drawn from the population.
\begin{problem}{e} For this part only, suppose $Y_{i 1}-Y_{i 0}$ equals a constant $c$. Is the slope estimator from an OLS regression of $Y_{i}$ on $\left(1, D_{i}\right)$ consistent for $c$ ? Make sure to derive the limit of this estimator, and then argue whether this limit equals $c$. Offer some intuition behind your results. 
\end{problem}

This is simple. We did this with Azeem. I'll type up tomorrow. \begin{problem}{f}Suppose that treatment $D_{i}$ is randomized so that $P\left(D_{i}=1\right)=0.5$. Show that $\kappa=$ ATE.
\end{problem}

Our parameter is 

\begin{align*}
\kappa & :=E\left[Y_{1i}|D_{i}=1\right]-E\left[Y_{0i}|D_{i}=0\right]\\
 & =E\left[Y_{1i}\right]-E\left[Y_{0i}\right]\\
 & =E\left[Y_{1i}-Y_{0i}\right]
\end{align*}

where the second equality comes from random assignment. The third
equality comes from the linearity of expectations. 




\begin{enumerate}[(a), wide, labelwidth=!, labelindent=0pt]
    \item \textbf{}
\end{enumerate}

\newpage
%%%%%%%%%%%%%%%%%%%%%%%%%%%%%%%%%% QUESTION 3 %%%%%%%%%%%%%%%%%%%%%%%%%%%%%%%%%
\section*{Problem 2}

\textbf{}
\begin{enumerate}[(a), wide, labelwidth=!, labelindent=0pt]
    \item \textbf{}
\end{enumerate}

\end{document}

\newpage
%%%%%%%%%%%%%%%%%%%%%%%%%%%%%%%%%% QUESTION 3 %%%%%%%%%%%%%%%%%%%%%%%%%%%%%%%%%
\section*{Problem 2}

\textbf{}
\begin{enumerate}[(a), wide, labelwidth=!, labelindent=0pt]
    \item \textbf{}
\end{enumerate}

\end{document}
