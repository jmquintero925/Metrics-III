\documentclass{article}

\usepackage[margin=1in]{geometry} 
\usepackage{amsmath,amsthm,amssymb,amsfonts, fancyhdr, color, comment, graphicx, environ,bbm}
\usepackage[dvipsnames]{xcolor}
\usepackage{subcaption}
\usepackage{mdframed}
\usepackage[shortlabels]{enumitem}
\usepackage{indentfirst}
\usepackage{hyperref}
\usepackage{placeins}
\usepackage{comment} %Comment large blocks
\usepackage{xfrac}
\usepackage{float} % To use "H" to force tables to be where wanted
\usepackage{booktabs} % Makes output from knittr:kable look better
\usepackage{parskip} %No indents for paragraphs
\usepackage[shortlabels]{enumitem} % To enumerate with letters
\hypersetup{
 colorlinks=true,
 linkcolor=blue,
 filecolor=magenta, 
 urlcolor=blue,
}
\pagestyle{fancy}
\usepackage{todonotes}

\newcommand{\E}{\mathbb{E}}
\renewcommand{\H}{\mathcal{H}}
\renewcommand{\L}{\mathcal{L}}
\newcommand{\dU}[1]{\ensuremath\frac{\partial u}{\partial #1}}
\newcommand{\dV}[1]{\ensuremath\frac{\partial v}{\partial #1}}
\newcommand{\ppx}[2]{\ensuremath\frac{\partial #1}{\partial #2}}
\newcommand{\ddx}[2]{\ensuremath\frac{d #1}{d #2}}
\newcommand{\indp}{\perp\!\!\!\perp} 

%\usepackage{mathpazo} %Dylan's fancy math bullshit that I don't want
\usepackage{microtype}
\usepackage{graphicx}
\usepackage{setspace}

%Footnote without a number
\newcommand\blfootnote[1]{%
  \renewcommand\thefootnote{}\footnote{#1}%
  \addtocounter{footnote}{-1}%
}

% Problem formatting [Alex]

\newenvironment{problem}[1]
    { \begin{mdframed}[backgroundcolor=Periwinkle!20] \textbf{(#1)} }
    {  \end{mdframed}}
% Define solution environment
\newenvironment{solution}{\textbf{Solution}\\}

%%%%%%%%%%%%%%%%%%%%%%%%%%%%%%%%%%%%%%%%%%%%%
%Fill in the appropriate information below
\lhead{Problem Set 1}
\rhead{Empirical Analysis} 
\title{Problem Set 3}
\author{Alex Weinberg \and Isaac Norwich \and Jose M. Quintero}

%%%%%%%%%%%%%%%%%%%%%%%%%%%%%%%%%%%%%%%%%%%%%
\begin{document}
\maketitle

Our code can be found in this GitHub respository: \url{https://github.com/jmquintero925/Metrics-III/tree/main/ps3}

%Q1 - Alex
%Q2 - Isaac
%Q3 - Jose
%Q4 - Isaac


%%%%%%%%%%%%%%%%%%%%%%%%%%%%%%%%%% QUESTION 1 %%%%%%%%%%%%%%%%%%%%%%%%%%%%%%%%%
\section*{Problem 1}
Suppose you want to understand how neighborhoods affect children's
future earnings. You collect an i.i.d. sample $\left\{ Y_{i},D_{i}\right\} _{i=1}^{n}$,
where $Y_{i}$ is a child's income in adulthood and $D_{i}$ indicates
whether that child grows up in a low-poverty neighborhood $(1=yes$
, $0=no)$. 

% ------------------------------------
\begin{problem}{a}
Define the ATE, ATT, and ATUT in this setting. When can you estimate
them via OLS? Do you believe that these restrictions you impose are
likely to hold? Briefly explain. 
\end{problem}

Yes, I define them as follows
\begin{align*}
ATE & :=E\left[Y_{1}-Y_{0}\right]\\
ATT & :=E\left[Y_{1}-Y_{0}|D=1\right]\\
ATU & :=E\left[Y_{1}-Y_{0}|D=0\right]
\end{align*}

These are hard to estimate using OLS because OLS is going to pick
up on selection bias. I decompose the OLS estimand for with respect
to the ATT but the idea is the same for ATU, ATE, etc. 
\begin{align*}
\beta^{OLS} & =\frac{Cov\left[Y,D\right]}{Var\left[D\right]}\\
 & =E\left[Y_{1}|D=1\right]-E\left[Y_{0}|D=0\right]\\
 & =\underbrace{E\left[Y_{1}|D=1\right]-E\left[Y_{0}|D=1\right]}_{ATT}+\underbrace{E\left[Y_{0}|D=1\right]-E\left[Y_{0}|D=0\right]}_{selection}
\end{align*}

However if $D\perp Y_{1},Y_{0}$ then we have treatment randomly assigned
so then $E\left[Y_{0}|D=1\right]-E\left[Y_{0}|D=0\right]=0$ and we
are good. OLS recovers $ATE=ATT=ATU$.

\rule[0.5ex]{1\columnwidth}{1pt}

% ------------------------------------
\begin{problem}{b}
Suppose you collect data about $X_{i}$, the parents' annual incomes
when the child is young. 
\end{problem}


(i) Under selection-on-observables (i.e. $Y_{d,i}\perp D_{i}\mid X_{i}$
), explain how to estimate the ATE. 

\rule[0.5ex]{1\columnwidth}{1pt}

If I assume that $Y_{d,i}\perp D_{i}\mid X_{i}$ then I define the
ATE and use the independence assumption to get
\begin{align*}
ATE & :=E\left[E\left[Y_{1}|X\right]-E\left[Y_{0}|X\right]\right]\\
 & =E\left[E\left[Y_{1}|D=1,X\right]-E\left[Y_{0}|D=0,X\right]\right]
\end{align*}

(ii) Do you believe selection-on-observables is likely to hold in
this case? Briefly explain. 

\rule[0.5ex]{1\columnwidth}{1pt}

Selection on observables seems hard to justify here. The assumption
requires tht 
\[
E\left[Y_{d}|D=0,X\right]=E\left[Y_{d}|D=1,X\right]\qquad\forall d,X
\]

In words, in our context, that says \emph{conditional on parental
income, children in poor neighborhoods would have had the same outcomes
as children who grew up in rich neighborhoods. }I further assume that
there is enough support (i.e. we observe rich families in poor neighborhoods). 
\begin{itemize}
\item This assumption fails because I believe parental income is an outcome
variable. If one of the mechanisms by which bad neighborhoods hurt
kids is by reducing parental income (e.g by poor access to jobs) then
we are controlling for a mechanism - not kosher.
\item This fails also because there is unobserved selection bias in the
parents who choose to live in different neighborhoods. 
\begin{itemize}
\item Consider poor parents who choose to live in rich neighborhoods because
they are very excited about the good schools available for their children
there. Those parents are likely also taking other steps to improve
their childs outcomes. So when we compare means we end up incorporating
selection bias (here parental quality). 
\end{itemize}
\end{itemize}

\begin{problem}{c}
Suppose there exists a program
that gives housing vouchers to underserved families with young children.
Each voucher subsidizes the cost of moving to a low-poverty neighborhood.
These vouchers are randomly assigned to families, and not everyone
who is offered a voucher accepts one. Let $Z_{i}$ indicate whether
a child's family receives a voucher $(1=yes,0=no)$. Assume that your
population consists of children whose families are eligible for vouchers. 
\end{problem}


(i) Define the compliers, always-takers, never-takers, and defiers
in this setting. 

\rule[0.5ex]{1\columnwidth}{1pt}

\begin{align*}
D_{1}=1, & D_{0}=0\tag{compliers}\\
D_{1}=0, & D_{0}=1\tag{defiers}\\
D_{1}=1, & D_{0}=1\tag{always takers}\\
D_{1}=0, & D_{0}=0\tag{never takers}
\end{align*}

In this setting Always-Takers move to rich neighborhoods regardless
of voucher. Never-takers similarly never move, regardless of voucher.
Compliers move if they get the voucher and don't move otherwise. Defiers
move if they don't get the voucher and don't move if they do. 

(ii) Argue whether $Z_{i}$ is a valid instrument for $D_{i}$. 

\rule[0.5ex]{1\columnwidth}{1pt}

We need four conditions to have a valid instrument. 
\begin{enumerate}
\item \textbf{Random assignment.} Satisfied here because voucher given by
lottery.
\[
Y_{d,z},D_{z}\perp Z
\]
\item \textbf{Exclusion.} I believe that winning the voucher lottery only
affects childrens outcomes via the treatment. I believe this because
the voucher size is likely small and does not create income effects.
\[
Y_{d,1}=Y_{d,0}\qquad\forall d
\]
\item \textbf{First stage.} This is testable. It seems likely the voucher
would have an effect on movement decisions if voucher size large enough.
\[
E\left[D_{1}-D_{0}\right]\neq0
\]
\item \textbf{Monotonicity (Uniformity)}. This seems sensible. I can't imagine
that people who receive a voucher are less likely to move. 
\[
D_{1}-D_{0}\geq0
\]
\end{enumerate}
\rule[0.5ex]{1\columnwidth}{1pt}

(iii) In general, the ATT and LATE
are not the same, since the ATT is a weighted average of two effects:
one on always-takers and one on compliers. Show that this is the case. 

\rule[0.5ex]{1\columnwidth}{1pt}

Recall the definition of the ATT. Use the law of total probability.

\begin{align*}
ATT & :=E\left[Y_{1}-Y_{0}|D=1\right]\\
 & =E\left[Y_{1}-Y_{0}|D_{1}>D_{0},D=1\right]Pr\left(D_{1}>D_{0}\right)+E\left[Y_{1}-Y_{0}|D_{1}=D_{0},D=1\right]Pr\left(D_{1}>D_{0}\right)\\
\end{align*}

We can see that the ATT is the weighted average of the treatment effect
for two groups. The first group $D_{1}>D_{0}$ are the compliers.
The second group $D_{1}=D_{0}=1$ are the always-takers. Note that
$D_{1}\geq D_{0}$ by the monotonicity assumption. 

\rule[0.5ex]{1\columnwidth}{1pt}

(iv) Assuming $Z_{i}$ is a valid
instrument for $D_{i}$, will running IV estimate the ATE? If so,
explain. If not, specify the exact conditions under which the LATE
equals the ATE. 

\rule[0.5ex]{1\columnwidth}{1pt}

No, IV will not estimate the ATE. It will estimate the LATE. We showed
this in class. If the entire population are compliers then 
\[
ATE=LATE
\]

I will do the decomposition to show this is true. 

If the $LATE_{always}=LATE_{complier}=LATE_{never}$ then LATE is
also ATE.

(v) Describe how to consistently estimate the average treatment effect
of $Z_{i}$ on $Y_{i}$.

\rule[0.5ex]{1\columnwidth}{1pt}

I want to estimate 
\[
ATE_{Z}=E\left[Y|Z=1\right]-E\left[Y|Z=0\right]
\]

Recall
\begin{align*}
Y & =Y_{1}D+Y_{0}\left(1-D\right)\\
D & =D_{1}Z+D_{0}\left(1-Z\right)
\end{align*}

Since $Z$ is randomly assigned we can run OLS. 
\begin{align*}
\beta^{OLS,Z} & =\frac{Cov\left[Y,Z\right]}{Var\left[Z\right]}\\
 & =\frac{E\left[YZ\right]-E\left[Y\right]E\left[Z\right]}{E\left[Z^{2}\right]-E\left[Z\right]^{2}}\\
 & =\frac{E\left[Y|Z=1\right]-E\left[Y|Z=0\right]}{E\left[D|Z=1\right]-E\left[D|Z=0\right]}
\end{align*}

I can write the denominator as
\begin{align*}
E\left[D|Z=1\right]-E\left[D|Z=0\right] & =E\left[D_{1}|Z=1\right]-E\left[D_{0}|Z=0\right]\\
 & =E\left[D_{1}-D_{0}\right]\tag{random assignment}\\
 & =E\left[D_{1}-D_{0}|D_{1}>D_{0}\right]Pr\left(D_{1}>D_{0}\right) \\
 &+E\left[D_{1}-D_{0}|D_{1}=D_{0}\right]Pr\left(D_{1}=D_{0}\right)+E\left[D_{1}-D_{0}|D_{1}<D_{0}\right]Pr\left(D_{1}<D_{0}\right)\\
 & =E\left[D_{1}-D_{0}|D_{1}>D_{0}\right]Pr\left(D_{1}>D_{0}\right)\\
 & =Pr\left(D_{1}>D_{0}\right)
\end{align*}

where the always-takers, never-takers drop out because $D_{1}-D_{0}=0$
and defiers are ruled out by assumption $Pr\left(D_{1}<D_{0}\right)=0$.

The numerator becomes
\begin{align*}
E\left[Y|Z=1\right]-E\left[Y|Z=0\right] & =E\left[Y_{1}D+Y_{0}\left(1-D\right)|Z=1\right]-E\left[Y_{1}D+Y_{0}\left(1-D\right)|Z=0\right]\\
 & =E\left[Y_{1}\left(D_{1}Z+D_{0}\left(1-Z\right)\right)+Y_{0}\left(1-\left(D_{1}Z+D_{0}\left(1-Z\right)\right)\right)|Z=1\right] \\
 & -E\left[Y_{1}\left(D_{1}Z+D_{0}\left(1-Z\right)\right)+Y_{0}\left(1-\left(D_{1}Z+D_{0}\left(1-Z\right)\right)\right)|Z=0\right]\\
 & =E\left[Y_{1}\left(D_{1}Z\right)+Y_{0}\left(1-\left(D_{1}Z\right)\right)|Z=1\right]-E\left[Y_{1}\left(D_{0}\left(1-Z\right)\right)+Y_{0}\left(1-\left(D_{0}\left(1-Z\right)\right)\right)|Z=0\right]\\
 & =E\left[Y_{1}D_{1}+Y_{0}\left(1-D_{1}\right)|Z=1\right]-E\left[Y_{1}D_{0}+Y_{0}\left(1-D_{0}\right)|Z=0\right]\\
 & =E\left[Y_{1}D_{1}+Y_{0}\left(1-D_{1}\right)-Y_{1}D_{0}+Y_{0}\left(1-D_{0}\right)\right]\tag{random assignment}\\
 & =E\left[Y_{1}\left(D_{1}-D_{0}\right)-Y_{0}\left(D_{1}-D_{0}\right)\right]\\
 & =E\left[\left(Y_{1}-Y_{0}\right)\left(D_{1}-D_{0}\right)\right]\\
 & =E\left[\left(Y_{1}-Y_{0}\right)|\text{complier}\right]Pr\left(D_{1}>D_{0}\right)+0\times Pr\left(D_{1}=D_{0}\right)+E\left[\left(Y_{1}-Y_{0}\right)\right]Pr\left(D_{1}<D_{0}\right)\\
 & =E\left[\left(Y_{1}-Y_{0}\right)|\text{complier}\right]Pr\left(D_{1}>D_{0}\right)
\end{align*}

where for the last inequality always takers, never takers, and defiers
drop out. 

Divide numerator by denominator to get 
\begin{align*}
\beta^{IV} & =\frac{E\left[Y|Z=1\right]-E\left[Y|Z=0\right]}{E\left[D|Z=1\right]-E\left[D|Z=0\right]}\\
 & =\frac{E\left[\left(Y_{1}-Y_{0}\right)|\text{complier}\right]Pr\left(D_{1}>D_{0}\right)}{Pr\left(D_{1}>D_{0}\right)}\\
 & =E\left[\left(Y_{1}-Y_{0}\right)|\text{complier}\right]
\end{align*}


\newpage
%%%%%%%%%%%%%%%%%%%%%%%%%%%%%%%%%% QUESTION 2 %%%%%%%%%%%%%%%%%%%%%%%%%%%%%%%%%
\section*{Problem 2}
Setup text here

\begin{problem}{a}
\end{problem}
\begin{solution}
\end{solution}

\begin{problem}{b}
\end{problem}
\begin{solution}
\end{solution}

\begin{problem}{c}
\end{problem}
\begin{solution}
\end{solution}

\begin{problem}{d}
\end{problem}
\begin{solution}
\end{solution}

\begin{problem}{e}
\end{problem}
\begin{solution}
\end{solution}

\begin{problem}{f}
\end{problem}
\begin{solution}
\end{solution}

\newpage
%%%%%%%%%%%%%%%%%%%%%%%%%%%%%%%%%% QUESTION 3 %%%%%%%%%%%%%%%%%%%%%%%%%%%%%%%%%
\section*{Problem 3}
Setup text here

\begin{problem}{a}
\end{problem}
\begin{solution}
\end{solution}

\begin{problem}{b}
\end{problem}
\begin{solution}
\end{solution}

\begin{problem}{c}
\end{problem}
\begin{solution}
\end{solution}

\begin{problem}{d}
\end{problem}
\begin{solution}
\end{solution}

\newpage
%%%%%%%%%%%%%%%%%%%%%%%%%%%%%%%%%% QUESTION 4 %%%%%%%%%%%%%%%%%%%%%%%%%%%%%%%%%
\section*{Problem 4}
Setup text here

\begin{problem}{a}
\end{problem}
\begin{solution}
\end{solution}

\begin{problem}{b}
\end{problem}
\begin{solution}
\end{solution}

\begin{problem}{c}
\end{problem}
\begin{solution}
\end{solution}

\begin{problem}{d}
\end{problem}
\begin{solution}
\end{solution}

\begin{problem}{e}
\end{problem}
\begin{solution}
\end{solution}

\begin{problem}{f}
\end{problem}
\begin{solution}
\end{solution}

\begin{problem}{g}
\end{problem}
\begin{solution}
\end{solution}

\begin{problem}{h}
\end{problem}
\begin{solution}
\end{solution}

\end{document}
