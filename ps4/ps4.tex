\documentclass{article}

\usepackage[margin=1in]{geometry} 
\usepackage{amsmath,amsthm,amssymb,amsfonts, fancyhdr, color, comment, graphicx, environ,bbm}
\usepackage[dvipsnames]{xcolor}
\usepackage{subcaption}
\usepackage{mdframed}
\usepackage[shortlabels]{enumitem}
\usepackage{indentfirst}
\usepackage{hyperref}
\usepackage{placeins}
\usepackage{comment} %Comment large blocks
\usepackage{xfrac}
\usepackage{float} % To use "H" to force tables to be where wanted
\usepackage{booktabs} % Makes output from knittr:kable look better
\usepackage{parskip} %No indents for paragraphs
\usepackage[shortlabels]{enumitem} % To enumerate with letters
\hypersetup{
 colorlinks=true,
 linkcolor=blue,
 filecolor=magenta, 
 urlcolor=blue,
}
\pagestyle{fancy}
\usepackage{todonotes}

\newcommand{\E}{\mathbb{E}}
\renewcommand{\H}{\mathcal{H}}
\renewcommand{\L}{\mathcal{L}}
\newcommand{\dU}[1]{\ensuremath\frac{\partial u}{\partial #1}}
\newcommand{\dV}[1]{\ensuremath\frac{\partial v}{\partial #1}}
\newcommand{\ppx}[2]{\ensuremath\frac{\partial #1}{\partial #2}}
\newcommand{\ddx}[2]{\ensuremath\frac{d #1}{d #2}}
\newcommand{\indp}{\perp\!\!\!\perp} 

%\usepackage{mathpazo} %Dylan's fancy math bullshit that I don't want
\usepackage{microtype}
\usepackage{graphicx}
\usepackage{setspace}

%Footnote without a number
\newcommand\blfootnote[1]{%
  \renewcommand\thefootnote{}\footnote{#1}%
  \addtocounter{footnote}{-1}%
}

% Problem formatting [Alex]

\newenvironment{problem}[1]
    { \begin{mdframed}[backgroundcolor=Periwinkle!20] \textbf{(#1)} }
    {  \end{mdframed}}
% Define solution environment
\newenvironment{solution}{\textbf{Solution}\\}

%%%%%%%%%%%%%%%%%%%%%%%%%%%%%%%%%%%%%%%%%%%%%
%Fill in the appropriate information below
\lhead{Problem Set 1}
\rhead{Empirical Analysis} 
\title{Problem Set 4}
\author{Alex Weinberg \and Isaac Norwich \and Jose M. Quintero}

%%%%%%%%%%%%%%%%%%%%%%%%%%%%%%%%%%%%%%%%%%%%%
\begin{document}
\maketitle

Our code can be found in this GitHub respository: \url{https://github.com/jmquintero925/Metrics-III/tree/main/ps4}


%%%%%%%%%%%%%%%%%%%%%%%%%%%%%%%%%% QUESTION 1 %%%%%%%%%%%%%%%%%%%%%%%%%%%%%%%%%
\section*{Problem 1}
In addition to regular neighborhood schools which are open to all local students, the Chicago Public School System contains several elite ``selective enrollment'' high schools which a student must test into. Assume seats at selective enrollment schools are allocated purely on the basis of test scores\footnote{Note the actual process is more complicated.} and that there is full compliance - all students with test scores above a certain cutoff will be assigned a seat and will attend, and no students who didn't test above the threshold will attend.

\begin{problem}{1}
Suppose you are interested in the average effect on prime-age wages of attending a selective enrollment school for all students who actually did attend a selective enrollment school. Can you identify such an effect if you observed data on all Chicago public high schoolers, their test scores, their high school, and their wages at 40?
\end{problem}
\begin{solution}
No, we wouldn't be able to identify the average effect of attending a selective school for \textbf{all} students who actually did attend a selective enrollment school. Due to the test score threshold, treatment is correlated with underlying $Y_0$ and $Y_0$, so we cannot recover the ATE.
\end{solution}

\begin{problem}{2}
If not, what effect can you identify? How would you identify it? Is it interesting? Is it policy-relevant? For what policy?
\end{problem}
\begin{solution}
For simplicity, assume there is one public school and one selective public school.

With full compliance, we are in the sharp RD world. Let $R$ be the test score, our running variable, and $c$ be the threshold for getting in to selective public schools. That is, $Y=Y_1$ if $R\geq c$ and $Y=Y_0$ if $R<c$ where 1 denotes selective and 0 regular public schools. If we assume that $E[Y_d | R = r]$ is continuous at $r=c$ for $d \in \{0,1\}$ then we can identify the ATE at the cutoff:
\begin{equation*}
    E[Y_1=Y_0 | R=c] =  \lim_{r \downarrow c} E[Y|R=r] - \lim_{r \uparrow c} E[Y|R=r]
\end{equation*}
Note that with the continuity assumption, this becomes:
\begin{equation*}
    E[Y_1=Y_0 | R=c] =  E[Y_1|R=c] -  E[Y_0|R=c]
\end{equation*}

This treatment effect is interesting if we are considering policy choices where we would increase or decrease the number of spots at the selective schools. If we increase the number spot by one, the student who now gets in is expected to have the above treatment effect. If we decrease the number of spots by one, then the negative of this value is how that student who no longer goes to a selective school's income decreases.

\end{solution}

\begin{problem}{3}
Now suppose some high-schoolers in Chicago exit the Public School system and instead attend a Catholic or other private high school. How would this change your identification? Be specific about the new identification procedure and whether the treatment effect you're identifying in this part should be interpreted differently than in the previous part. Is this effect policy-relevant?
\end{problem}
\begin{solution}
If some high-schoolers exit the system and attend a catholic high school, the identification described above is no longer valid if we do not have random attrition. 

First, let's discuss how differential attrition impacts the LATE identified above. Since we no longer have perfect compliance (if we view compliance as graduating, not enrolling), then we have an instrument $Z$ that takes two values: $Z=1$ when $R\geqc$ is you get in to public school and $Z=0$ when $R<c$ is get in to public school. Students who attend public school have $D=0$, attend selective have $D=1$, and attend catholic/private school have $D=c4$. 
\begin{itemize}
    \item If students who have $R<c$ and thus are accepted to public school, $Z=0$, but attend private school, $D=c$, are more likely to have scores right below the threshold, $\operatorname{cov}(Y_0,D=c|Z=0)>0$, then the parameter we would estimate using the above formula will be too large: $\beta_RD > \beta_{LATE}$. This story is plausible: (rich) parents who think that think that their kid's $Y_c>Y_0$ would choose to enroll their child in private school. Since these children are more likely to have high $Y_0$'s compared to those with $R<c$, possibly due to their parent's high SES, the mass on the left hand side of the cutoff is lower than it should be will perfect compliance. 
    \item If going to catholic school only benefits certain types of people and those types do not have systematically higher or lower $Y_0$'s, then our estimate from above is still correct. That is, if attrition is orthogonal to $R$, $Y_0$, and $Y_1$, then our identification above is valid. If not, then we need a new identification strategy.
    \item Consider the case where students only learn their $Y_d$ for $d\in \{0,1\}$ once they attend either public or selective school. Students who go to regular schools don't learn anything about $Y_c$ because they don't interact with people who go to catholic schools. Students who attend selective schools do, and therefore learn what their $Y_c$ would be. For students with $R>c$, if $Y_c>Y_1$ then they leave. If this difference is negatively correlated with $Y_1$, then we end up with missing mass on the right hand side of the cutoff and $\beta_RD < \beta_{LATE}$.
\end{itemize}

With attrition, we now no longer have perfect compliance and instead can estimate the treatment effect using a fuzzy RD design. While it is no longer the case that $R>c$ is an exact predictor of $D$, it still may still be a strong predictor. We now think of $Z\equiv \mathbf{1}[R\geq c]$ as an incentive for $D$. We assume monotonicity, $D_1 \geq D_0$, leaving three types $T$: always-takers, compliers, and never-takers. We also assume that $\mathbb{E}\left[Y_{d} \mid R=r, T=t\right]$ and $\mathbb{P}[T=t \mid R=r]$ are continuous at $r=c$. In addition, I assume that $Y_c>Y_0$ so that students below the threshold who can afford private/catholic school will attend and be never-takers. With these assumptions, we estimate the LATE by taking the limit of the Wald estimate as $R \to c$:
\begin{align*}
    \frac{\lim _{r \downarrow c} \mathbb{E}[Y \mid R=r]-\lim _{r \uparrow c} \mathbb{E}[Y \mid R=r]}{\lim _{r \downarrow c} \mathbb{E}[D \mid R=r]-\lim _{r \uparrow c} \mathbb{E}[D \mid R=r]}=\mathbb{E}\left[Y_{1}-Y_{0} \mid R=c, Complier\right]
\end{align*}
Note the similar Wald estimand formula on the left side above. So what does this LATE tell us and how does it compare to the LATE in the first case? It's a very local parameter---meaning that it is not only the difference in $Y_1$ and $Y_0$ at $R=c$---but also only for those who are compliers at the cutoffs. So the people that this estimand is identified off of are different from the estimand in part 1. Whether it is policy relevant again depends on how interesting we think the instrument is, which in this case is the cutoff test score. If we lower the cutoff score and allow one more person in to selective high schools, we now have to consider whether that person is a complier or if they would instead go to catholic school (a never-taker). It's possible that the treatment effect for a complier is higher than for a never-taker at the cutoff, so the estimate is larger in magnitude than in the first part.

\end{solution}

\newpage
%%%%%%%%%%%%%%%%%%%%%%%%%%%%%%%%%% QUESTION 2 %%%%%%%%%%%%%%%%%%%%%%%%%%%%%%%%%
\section*{Problem 2}
Setup text here

\begin{problem}{a}
\end{problem}
\begin{solution}
\end{solution}

\begin{problem}{b}
\end{problem}
\begin{solution}
\end{solution}

\begin{problem}{c}
\end{problem}
\begin{solution}
\end{solution}

\begin{problem}{d}
\end{problem}
\begin{solution}
\end{solution}

\begin{problem}{e}
\end{problem}
\begin{solution}
\end{solution}

\begin{problem}{f}
\end{problem}
\begin{solution}
\end{solution}

\end{document}
